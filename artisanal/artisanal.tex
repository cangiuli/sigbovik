\documentclass[10pt]{article}
\usepackage[colorlinks=true,linkcolor=black,urlcolor=black,citecolor=black]%
{hyperref}
\usepackage{proof}
\usepackage{amsmath}
\usepackage{cite}
\usepackage{graphicx}
\usepackage{stmaryrd}

\title{Artisanal type theory}
\author{Carlo Angiuli}
\date{April 1, 2015}

\begin{document}
\maketitle

\section{A brief history of some things}

Food was invented by Mesopotamians some 5,000 years ago, and has been eaten ever
since. 
%
Logic was invented in the Mediterranean by ancient Greeks, including Aristotle
and a mortal man \cite{Aristotle40} named Socrates. It lives on as an important
course in pre-law curricula across the United States. 

Modern times require more modern logics. Computer programming is closely tied to
\emph{intuitionistic} logic, in which proofs of a proposition correspond
directly to algorithms. Intuitionistic logic, often in the form of type theory,
is taught to several American computer scientists annually.

Until the past several centuries, food and logic were primarily manufactured by
artisans, who trained apprentices in the arts of, respectively, proofing and
proving.
%
The Industrial Revolution gave rise to machines able to produce food and
textiles much faster than artisans ever could.
%
The digital revolution, likewise, has turned `computer' from a human job into a
cheap, ubiquitous machine capable of multiple calculations per second.

Despite the overwhelming success of mass-produced food, some consumers want to
revisit food's roots as a product sustainably and ethically crafted by local
artisans using traditional techniques. The result, known as \emph{artisanal
food}, has taken off in popularity in the past few years \cite{NYT09,Cope14}.

\section{Algorithms with the human touch}

Now that computers 

Algorithms have lost the personal touch. They can be evaluated

But intuitionistic logic, while about algorithms, need not be about computers.
We should return to the roots of computation---slow, error-prone, human 


Of course, since type theories simply describe constructive proofs, not computer
programs per se \cite{MartinLof85}

must write the derivation by hand

we maintain the invariant that closed terms evaluate to a normal form
accompanied by a certificate of authenticity that a human performed that
evaluation.

this fact demonstrates, among other things, a firm commitment to \emph{locally-}
grown, sound, and complete 

Since each term was lovingly handcrafted and normalized by a human, these
semantics provide a more \emph{meaningful} explanation of type theory than
traditional computer-based interpreters.






\section{Examples}

\subsection{Tiny data}

shallow learning

build a classifier for booleans

\subsection{Small-batch jobs}

\vspace{10em}

as a foundation of \emph{human} science

logic as if people (but not computers) matter
technology is ruining human interaction
computers are ruining mathematical interaction??

respect for employees

heirloom

\bibliography{artisanal}{}
\bibliographystyle{acm}

\end{document}
