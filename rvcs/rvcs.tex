\documentclass[10pt]{article}
\usepackage[colorlinks=true,linkcolor=black,urlcolor=black,citecolor=black]%
{hyperref}
\usepackage{graphicx}

\title{Redistributive version control systems}
\author{Karlo Angiuli \and Frederick Engels}

\begin{document}
\maketitle

\begin{abstract}\noindent
A spectre is haunting Github; the spectre of communism. Distributed version
control has led to a new era of open-source software, but commit inequality
remains rampant. We describe a system in which programmers code according to
their abilities, on repositories according to their needs. 
\end{abstract}

\section{Open-Source Bourgeois and Proletarians}

The history of all hitherto existing software is the history of class
struggles.\footnote{The history of all hitherto existing object-oriented
programming is the history of \texttt{class} struggles.}

iTunes and Winamp,
Borland C++ and Intel C++ Compiler,
Adobe Illustrator and Adobe FreeHand,
Microsoft Word and Corel WordPerfect,
in a word, oppressor and oppressed, stood in constant opposition to one another,
carried on an uninterrupted, now hidden, now open fight, a fight that each time
ended in the ruin of a contending software package.

In the closed-source epochs of history, we find almost everywhere a complicated
arrangement of software into various orders, a manifold gradation of popularity.
The modern open-source society that has sprouted from the ruins of closed-source
society has not done away with class antagonisms. It has but established new
forms of struggle in place of the old ones.

Our epoch, the epoch of the open-source, possesses, however, this distinct
feature: it has simplified class antagonisms. The open-source movement, during
its rule of scarce thirty years, has created more massive and more colossal
development forces than have all preceding generations together.

But the open-source bourgeoisie have called into existence the coders who are to
bring their own demise---the open-source proletariat, the developers of
unsuccessful open-source software.

The lower strata of the open-source---the GNU Hurd, GNU arch, and Guile---all
these sink gradually into failure, partly because their diminutive development
activity does not suffice for the scale on which Modern Software, like Linux,
git, and Emacs Lisp, are developed.

With their failure begins their struggle with the open-source bourgeoisie. At
first the contest is carried on by individual developers, then by online
communities, against the individual software projects which succeed them. 

The proletariat direct their attacks not against the open-source conditions of
development, but against the other projects themselves; they argue on Internet
forums, they complain on listservs, they seek to restore by force the vanished
status of the developer of Lesser-Known Software, the Software Proletariat.

At this stage, the developers still form an incoherent mass scattered over the
whole Internet, broken up by their mutual competition. But with the development
of social version control hosts, the developers not only increase in number;
they become concentrated in greater masses. Thereupon, the developers begin to
form communities.

Altogether collisions between the classes of the old society further, in many
ways, the course of development of the open-source proletariat. The open-source
bourgeoisie finds itself involved in a constant battle. At first with the
closed-source community; later on, with those portions of the open-source
bourgeoisie itself, whose interests have become antagonistic to the progress of
software.

In all these battles, it sees itself compelled to appeal to the open-source
community at large. The bourgeoisie itself, therefore, supplies the proletariat
with its own elements of computer science and social education, in other words,
it furnishes the proletariat with knowledge for overcoming the bourgeoisie. What
the bourgeoisie therefore produces, above all, are its own grave-diggers. Its
fall and the victory of the proletariat are equally inevitable.

\section{Introduction}

30\% of your commits by line have to go to others' repositories, thanks to
socialist Obama.

\begin{quote}
Somebody invested in libraries and runtimes. If you've got a program---you
didn't build that. Somebody else's compiler made that happen.
\end{quote}

\section{Implementation issues}

\subsection{Commit avoidance}

Workarounds include maintaining a series of shell repositories, etc

Double Irish

Romney jokes

\pagebreak
\section*{Acknowledgements}
\includegraphics[width=4in]{thanks.png}

%\bibliography{citations}{}
%\bibliographystyle{acm}
\end{document}

% failed OSS projects:
% Plan 9? Gnash?
