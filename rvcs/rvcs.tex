\documentclass[10pt]{article}
\usepackage[colorlinks=true,linkcolor=black,urlcolor=black,citecolor=black]%
{hyperref}
\usepackage{graphicx}

\title{Redistributive version control systems}
\author{Karlo Angiuli \and Frederick Engels}

\begin{document}
\maketitle

\begin{abstract}\noindent
A spectre is haunting Github; the spectre of communism. Distributed version
control has led to a new era of free software, but commit inequality remains
rampant. We describe a system in which programmers code according to their
abilities, on repositories according to their needs. 
\end{abstract}

\section{Open-source bourgeois and proletarians}

\textbf{The history of all hitherto existing software is the history of class
struggles.}\footnote{The history of all hitherto existing object-oriented
programming is the history of \texttt{class} struggles.}

iTunes and Winamp,
Intel C++ Compiler and Borland C++,
Adobe Illustrator and Adobe FreeHand,
Microsoft Word and Corel WordPerfect,
in a word, \textbf{oppressor and oppressed}, stood in constant opposition to one
another, carried on an uninterrupted, now hidden, now open fight, a fight that
each time ended in the ruin of a contending software package.

In the closed-source epochs of history, we find almost everywhere a complicated
arrangement of software into various orders, a manifold gradation of popularity.
The modern open-source society that has sprouted from the ruins of closed-source
society has not done away with class antagonisms. It has but established new
forms of struggle in place of the old ones.

Our epoch, \textbf{the epoch of free software}, possesses, however, this
distinct feature: it has simplified class antagonisms. The free software
movement, during its rule of scarce thirty years, has created more massive and
more colossal development forces than have all preceding generations together.

But the successful projects, the \textbf{Open-Source Bourgeoisie}, have called
into existence the coders who are to bring their own demise---the \textbf{Free
Software Proletariat}, the developers of unsuccessful free software.

The lower strata of free software---the GNU Hurd, GNU arch, and Guile---all
these sink gradually into failure, partly because \textbf{their diminutive
development activity does not suffice for the scale} on which Modern Software,
like Linux, git, and Emacs Lisp, are developed.

With their failure begins their struggle with the open-source bourgeoisie. At
first the contest is carried on by individual developers, then by online
communities, against the individual software projects which succeed them. 

The proletariat direct their attacks not against the open-source conditions of
development, but against the other projects themselves; they \textbf{argue on
Internet forums}, they \textbf{complain on listservs}, they seek to restore by
force the vanished status of the developer of Lesser-Known Software, the
Software Proletariat.

At this stage, the developers still form an incoherent mass scattered over the
whole Internet, broken up by their mutual competition. But with the development
of social version control, the developers not only increase in number; they
become concentrated in greater masses. Thereupon, the developers begin to form
communities.

Altogether collisions between the classes of the old society further, in many
ways, the course of development of the open-source proletariat. \textbf{The
open-source bourgeoisie finds itself involved in a constant battle.} At first
with the closed-source community; later on, with those portions of the
open-source bourgeoisie itself, whose interests have become antagonistic to the
progress of software.

In all these battles, it sees itself compelled to appeal to the free software
community at large. The bourgeoisie itself, therefore, supplies the proletariat
with its own elements of computer science and social education, in other words,
it \textbf{furnishes the proletariat with knowledge for overcoming the
bourgeoisie}. What the bourgeoisie therefore produces, above all, are its own
grave-diggers. \textbf{Its fall and the victory of the proletariat are equally
inevitable.}

\section{Redistributive version control}

The immediate aim of \textbf{redistributive version control systems} (RVCS) is
formation of all open-source developers into a class, overthrow of the
open-source bourgeois supremacy, \textbf{conquest of all software development}
by the proletariat.

The Open-Source Revolution overthrew proprietary software in favor of free
software. The distinguishing feature of RVCS is not the abolition of free
software development generally, but the abolition of bourgeois free software
projects. But modern bourgeois software is the final and most complete
expression of the free software movement. In this sense, the theory of RVCS may
be summed up in the single sentence: \textbf{Abolition of anomalously successful
software.}

Specifically, the RVCS project aims:
\begin{enumerate}
\item To abolish all other version control.
\item To require that at least \textbf{30\% of each developer's commits}, by
diff line, must be to less fortunate repositories than one's own.
\item To displace the free software bourgeoisie, the Torvaldses and
Shuttleworths of the world, from their dominion over successful software
projects.
\end{enumerate}

Under an RVCS software development paradigm, the community will choose as one
the projects worthy of Communal development. Developer person-hours will be
allotted to each project according to need, from each developer according to
their ability.

% TODO more details?

\begin{quote}
Somebody invested in libraries and runtimes. \textbf{If you've got a program,
you didn't build that.} Somebody else's compiler made that happen.%
\hfill ---Barack ``Redistribution'' Obama
\end{quote}

\section{Implementation issues}

We acknowledge that, even in a RVCS paradigm, the bourgeoisie may wish to steal
for themselves the contribution that they owe the developers of the world. Such
\textbf{redistribution avoidance} will take many forms.

Such developers might 
\begin{enumerate}
\item establish their repositories in offshore jurisdictions (GitHub:Monaco,
Bitseau de Seychelles);
\item perpetually fork their own repositories, or set up shell repositories, in
order to disguise their own commit history;
\item avoid newlines in their own repositories, artificially decreasing their
perceived net worth; or
\item secret their repositories on hard drives, distributing releases via
carrier pigeon.
\end{enumerate}

\section*{Acknowledgements}
\includegraphics[width=4in]{thanks.png}

%\bibliography{citations}{}
%\bibliographystyle{acm}
\end{document}

% failed OSS projects:
% Plan 9? Gnash?
