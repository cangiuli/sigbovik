\documentclass[10pt]{article}
\usepackage[colorlinks=true,linkcolor=black,urlcolor=black,citecolor=black]%
{hyperref}
\usepackage{graphicx}
%\usepackage[T1]{fontenc}

\newcommand\tableflip{\includegraphics[height=.8\baselineskip]{tableflip.pdf}}
\newcommand\J[1]{\tableflip\rotatebox[origin=c]{180}{#1}}
%\newcommand\shrug{\texttt{\textasciitilde\textbackslash(${}^\circ\omega^\circ$)/\textasciitilde}}

\title{The $\infty$-accidentopos model of\\ unintentional type theory}
\author{Carlo Angiuli}

\begin{document}
\maketitle

\section{Introduction}

Unintentional type theory is a variant of Martin-L\" of type theory which serves
as an internal language of $\infty$-accidentoposes.

%Weak $\infty$-Groupoid Martin-L\" of type theory
\section{Syntax}

The ``J'' rule
\J{argument}











\pagebreak
\section*{Acknowledgements}

Thanks to Chris Martens for suggesting that I study UTT, and the Univalent
Foundations project for making it seem like a wise idea.

%\bibliography{citations}{}
%\bibliographystyle{acm}

\end{document}
